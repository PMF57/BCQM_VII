```latex
\documentclass[11pt,a4paper]{article}

\usepackage[margin=1in]{geometry}
\usepackage{hyperref}
\usepackage{booktabs}
\usepackage{enumitem}
\usepackage{amsmath}
\usepackage{xcolor}
\usepackage{framed}

\title{BCQM VI Path A Validation: Test Plan Beyond Stability}
\author{BCQM Research Notes}
\date{31 January 2026 (Revised)}

\begin{document}

\maketitle

\section*{Executive Summary}

Gate 1 (stability) has been cleared: community partitions show strong agreement (NMI $\sim$0.83--0.84, ARI $\sim$0.54--0.64), and super-graph ball-growth curves are stable ($d_{L2} \sim 10^{-2}$). This document outlines the validation tests for Gates 2--4 to establish that the super-graph is a genuine geometric object with physical significance.

\section*{Overview: Validation Gates}

\begin{itemize}
\item \textbf{Gate 1 (Stability):} PASSED --- Community structure and super-graph diagnostics are stable across seeds
\item \textbf{Gate 2 (Geometry):} Does the super-graph have intrinsic geometric structure?
\item \textbf{Gate 3 (Physics):} Does the super-graph connect to physical observables (threads, interactions)?
\item \textbf{Gate 4 (Scaling):} Do geometric properties converge and scale correctly with $N$?
\end{itemize}

\section{Gate 2: The Super-Graph Has Geometric Structure}

\subsection{Test 2.1: Spectral Dimension}

\subsubsection*{What}
Calculate the spectral dimension $d_s$ of the super-graph from the return probability of a random walk.

\subsubsection*{Method}
\begin{enumerate}
\item Run random walks on super-graph nodes
\item Measure return probability $P(t)$ at time $t$
\item Plot $\log P(t)$ vs $\log t$
\item Spectral dimension: $d_s = -2 \times \text{slope}$
\end{enumerate}

\subsubsection*{Why It Matters}
Real emergent geometries have finite $d_s$ even when the graph is topologically complex. CDT and other quantum gravity models predict $d_s \approx 2$ at short scales, $d_s \approx 4$ at long scales. If the super-graph shows $d_s \approx 2$--4, that is a strong hint of emergent spacetime structure.

\subsubsection*{Pass Criterion}
$d_s$ is finite, stable across seeds (standard deviation $<10\%$ of mean), and in physically plausible range (1.5--4.0).

\subsubsection*{Implementation Notes}
\begin{itemize}
\item Use 1000+ random walk samples per super-graph
\item Sample return probability at $t \in \{1, 2, 4, 8, 16, 32, \ldots\}$ up to $t_{\max} \sim K/2$
\item Average $d_s$ across all seeds at each $(N, n)$ configuration
\end{itemize}

\begin{framed}
\textbf{⚠ Finite-Size Warning:} Spectral dimension extraction suffers severe finite-size effects for $K < 50$ nodes. Random walks hit graph boundaries before establishing a clean power-law regime. At $N=8$ ($K \approx 22$--24), $d_s$ estimates will be noisy and may not be physically meaningful. Do not treat poor $d_s$ results at small $N$ as evidence of geometry failure. Rely on Test 2.4 (ball-growth dimension) for Phase 1 validation. Consider $d_s$ reliable only at $N \geq 32$ ($K \geq 50$).
\end{framed}

\subsection{Test 2.2: Graph Distance as Proto-Metric}

\subsubsection*{What}
Treat graph distance on the super-graph as a discrete metric; test metric axioms.

\subsubsection*{Method}
\begin{enumerate}
\item For all node triples $(i,j,k)$ in super-graph, check triangle inequality: $d(i,k) \leq d(i,j) + d(j,k)$
\item Compute ``metric-ness'' fraction: $\eta = \frac{\text{\# triples satisfying triangle inequality}}{\text{total \# triples}}$
\item Check symmetry: $d(i,j) = d(j,i)$ (should be automatic for undirected graphs)
\end{enumerate}

\subsubsection*{Why It Matters}
If the super-graph is emergent ``cloth'', it should have metric-like structure. Pure random graphs do not reliably satisfy the triangle inequality. Geometric graphs embedded in metric spaces do.

\subsubsection*{Pass Criterion}
$\eta > 0.95$ (allows small violations due to stochasticity). Stable across seeds.

\subsubsection*{Implementation Notes}
\begin{itemize}
\item For large $K$, sample $\sim$10,000 random triples rather than exhaustive enumeration
\item Report mean $\eta$ and standard deviation across seeds
\end{itemize}

\begin{framed}
\textbf{⚠ Methodological Warning:} This test is \textbf{tautological} if $d(i,j)$ is defined as unweighted shortest-path (hop) distance on the graph. Shortest paths satisfy the triangle inequality by construction, yielding $\eta = 1.0$ trivially. This test is only non-trivial if distance is defined from \textbf{inverse edge weights}: $d_{ij} = 1/w_{ij}$ or $d_{ij} = -\log w_{ij}$, then computing weighted shortest paths. Recommendation: \textbf{Defer this test to Phase 3 and merge with Test 3.3 (Metric Reconstruction)} where weighted distances are already being computed.
\end{framed}

\subsection{Test 2.3: Curvature Distribution}

\subsubsection*{What}
Assign Ollivier-Ricci curvature (or Forman curvature) to super-graph edges.

\subsubsection*{Method}
\begin{enumerate}
\item Use network geometry libraries (e.g., \texttt{GraphRicciCurvature} Python package)
\item Compute curvature for each edge in super-graph
\item Plot curvature distribution histogram
\item Check if distribution shape is stable across seeds at same $(N,n)$
\end{enumerate}

\subsubsection*{Why It Matters}
Emergent geometries show non-trivial curvature distributions. Flat space produces a narrow distribution near zero. Curved space produces a broad distribution, possibly bimodal. If Path A creates ``cloth'', systematic curvature at community boundaries may emerge.

\subsubsection*{Pass Criterion}
Non-degenerate curvature distribution (not all edges near zero), stable shape across seeds (KS test $p > 0.05$ for pairwise comparisons), possibly structured (not purely random).

\subsubsection*{Implementation Notes}
\begin{itemize}
\item Ollivier-Ricci curvature computation can be slow for large graphs; consider sampling edges
\item Report mean, median, and standard deviation of curvature
\item Overlay histograms from all seeds at same $(N, n)$ to check stability visually
\end{itemize}

\subsection{Test 2.4: Effective Dimensionality from Ball Growth}

\subsubsection*{What}
Extract effective dimension from ball-growth curves already computed.

\subsubsection*{Method}
\begin{enumerate}
\item At radius $r$, count $N(r)$ = nodes within graph distance $r$ from a random source node
\item In $d$-dimensional space: $N(r) \sim r^d$
\item Fit $\log N(r)$ vs $\log r$ in the linear regime, extract slope $= d_{\text{eff}}$
\item Average over multiple source nodes and across seeds
\end{enumerate}

\subsubsection*{Why It Matters}
This gives the ``ambient dimension'' the super-graph acts like it lives in. If $d_{\text{eff}} \approx 2$--4, that is physically suggestive (spacetime-like). If $d_{\text{eff}}$ is stable across seeds but varies with $n$, that maps connectivity strength to geometric dimensionality.

\subsubsection*{Pass Criterion}
$d_{\text{eff}}$ is finite, stable (standard deviation $<15\%$ of mean), and in physically plausible range (1.5--4.5).

\subsubsection*{Implementation Notes}
\begin{itemize}
\item Use ball-growth data already generated in stability tests
\item Fit in range $r \in [2, K/4]$ to avoid finite-size effects
\item Report $R^2$ of fit to assess quality of power-law scaling
\end{itemize}

\section{Gate 3: Super-Graph Connects to Physical Observables}

\subsection{Test 3.1: Thread Bundles as ``Particles'' on the Cloth}

\subsubsection*{What}
Threads (sequences of events from BCQM IV/V) should map to trajectories on the super-graph that are local in the emergent geometry.

\subsubsection*{Method}
\begin{enumerate}
\item Take a thread (sequence of events)
\item Map each event to its community $\rightarrow$ sequence of super-graph nodes
\item Check if thread-induced paths on super-graph are ``local'' (consecutive super-graph nodes are adjacent or close)
\item Compute ``localization index'':
\[
\text{LOC} = \frac{\text{\# thread steps with super-graph hop distance} \leq 1}{\text{total \# thread steps}}
\]
\item Compare to random baseline: shuffle community assignments and recompute LOC
\end{enumerate}

\subsubsection*{Why It Matters}
\textbf{This is the most important test in the validation plan.} If threads are physical (particle worldlines), they should be local on the emergent geometry. Random threads would hop randomly across the super-graph. Physical threads should trace out geodesic-like paths on the cloth. \textbf{This test directly establishes whether particles ``live on'' the emergent spacetime}---the core compatibility between quantum mechanics (threads) and general relativity (super-graph geometry).

\subsubsection*{Pass Criterion}
Threads show strong localization: $\text{LOC} > 0.70$, and significantly above random baseline ($p < 0.001$ in permutation test).

\subsubsection*{Implementation Notes}
\begin{itemize}
\item Identify threads using BCQM V thread-formation algorithm
\item For each thread, record sequence of communities visited
\item Compute super-graph hop distance between consecutive community pairs
\item Generate 1000 random permutations of community labels to establish baseline
\end{itemize}

\subsection{Test 3.2: Cross-Link Density as ``Interaction Strength''}

\subsubsection*{What}
Super-graph edge weights (number of Path A cross-links between communities) should correlate with physical interaction between threads in those communities.

\subsubsection*{Method}
\begin{enumerate}
\item Identify ``interacting'' threads: threads that share events or are causally linked
\item For each thread pair, record their home communities $(C_i, C_j)$
\item Check if these communities are connected in super-graph, and record edge weight $w_{ij}$
\item Compute correlation (Spearman $\rho$) between super-graph edge weight and thread interaction frequency
\end{enumerate}

\subsubsection*{Why It Matters}
If the super-graph encodes spatial proximity, then threads in nearby communities (high $w_{ij}$) should interact more frequently. This tests whether cloth-level adjacency $\rightarrow$ micro-level interaction.

\subsubsection*{Pass Criterion}
Significant positive correlation: Spearman $\rho > 0.5$, $p < 0.01$.

\subsubsection*{Implementation Notes}
\begin{itemize}
\item Define ``interaction'': threads that share $\geq 1$ event, or have causal links (event from thread A in causal past of event from thread B)
\item Aggregate interaction counts across all thread pairs in same community pair
\item Report scatterplot: $x = w_{ij}$, $y = \text{interaction count}$
\end{itemize}

\subsection{Test 3.3: Metric Reconstruction from Cross-Link Data}

\subsubsection*{What}
Use super-graph edge weights to define a proto-metric $g_{\mu\nu}$ on the cloth via metric embedding.

\subsubsection*{Method}
\begin{enumerate}
\item Convert edge weight $w_{ij}$ to distance: $d_{ij} = 1/w_{ij}$ (or $d_{ij} = -\log w_{ij}$)
\item Use metric embedding (MDS, Isomap, or force-directed layout) to assign coordinates $\{x_i\}$ to communities
\item Check embedding quality: stress (residual error), embedding dimension $d_{\text{embed}}$
\item Test if resulting metric has sensible properties: triangle inequality, smoothness
\end{enumerate}

\subsubsection*{Why It Matters}
This is the ``metric extraction'' step needed for BCQM VII/VIII. If Path A cross-links encode spatial proximity, this should produce a consistent metric. If it fails, Path A might not be encoding geometry correctly.

\subsubsection*{Pass Criterion}
\begin{itemize}
\item Metric embedding produces stable coordinates: stress $< 0.15$
\item Embedding dimension $d_{\text{embed}} \leq 4$
\item Coordinates stable across seeds (Procrustes alignment shows $>0.8$ similarity)
\item Reconstructed metric approximately satisfies triangle inequality ($\eta > 0.90$)
\end{itemize}

\subsubsection*{Implementation Notes}
\begin{itemize}
\item Use \texttt{scikit-learn.manifold.MDS} or similar
\item Try embedding dimensions $d \in \{2, 3, 4\}$, report which gives lowest stress
\item For stability check: embed super-graphs from different seeds, align via Procrustes, compute coordinate RMSD
\item \textbf{Note:} This test subsumes the non-trivial version of Test 2.2 (weighted triangle inequality)
\end{itemize}

\section{Gate 4: Scaling Behavior with $N$}

\subsection{Test 4.1: Community Count Scaling}

\subsubsection*{What}
Determine how $K$ (community count) scales with $N$ at fixed $n$.

\subsubsection*{Method}
\begin{enumerate}
\item Plot $K$ vs $N$ for $N \in \{4, 8, 16, 24, 32, 48, 64, 96, 128\}$ at fixed $n \in \{0.6, 0.8\}$
\item Fit power law: $K = A \cdot N^\alpha$
\item Extract scaling exponent $\alpha$ and goodness-of-fit $R^2$
\end{enumerate}

\subsubsection*{Why It Matters}
If the super-graph is a ``finite-size cloth'', expect $K \sim N^\alpha$ with $\alpha < 1$ (sublinear). If every event is its own community, $K \sim N$ (bad---no coarse-graining). If communities merge well, expect $\alpha \approx 0.5$--0.7 (good coarse-graining).

\subsubsection*{Pass Criterion}
$K \sim N^\alpha$ with $\alpha < 0.8$, stable fit ($R^2 > 0.95$) across $n$ regimes.

\subsubsection*{Implementation Notes}
\begin{itemize}
\item Average $K$ across seeds at each $N$
\item Fit on log-log plot: $\log K$ vs $\log N$
\item Report separate $\alpha$ for each $n$ regime
\end{itemize}

\subsection{Test 4.2: Spectral Dimension Convergence}

\subsubsection*{What}
Measure $d_s$ at multiple $N$, check for convergence as $N \to \infty$.

\subsubsection*{Method}
\begin{enumerate}
\item Compute $d_s$ (from Test 2.1) at each $N \in \{4, 8, 16, 32, 64, 128\}$
\item Plot $d_s(N)$ vs $N$
\item Fit to asymptotic form: $d_s(N) = d_s^\infty + A/N^\beta$
\item Check if $d_s$ flattens (derivative $\to 0$) by $N=128$
\end{enumerate}

\subsubsection*{Why It Matters}
True emergent geometry should have $d_s$ that stabilizes at large $N$. If $d_s$ keeps drifting, the super-graph might be a system-size artifact rather than a genuine geometric object.

\subsubsection*{Pass Criterion}
$d_s(N)$ flattens: $|d_s(128) - d_s(64)| < 0.2$. Converges to asymptotic value $d_s^\infty \in [1.5, 4.5]$.

\subsubsection*{Implementation Notes}
\begin{itemize}
\item Use same random walk protocol at all $N$
\item Report error bars (standard deviation across seeds) at each $N$
\end{itemize}

\subsection{Test 4.3: Super-Graph Diameter Scaling}

\subsubsection*{What}
Measure graph diameter $D_{\max}$ of super-graph giant component as function of $N$.

\subsubsection*{Method}
\begin{enumerate}
\item For each $N$, compute diameter $D_{\max}$ = longest shortest path in super-graph
\item Plot $\log D_{\max}$ vs $\log N$ (or $\log K$)
\item Fit scaling law: $D_{\max} \sim N^\gamma$ or $D_{\max} \sim \log N$
\end{enumerate}

\subsubsection*{Why It Matters}
Expected scaling laws:
\begin{itemize}
\item Small-world network: $D_{\max} \sim \log K \sim \log N$
\item Spatial network in $d$ dimensions: $D_{\max} \sim N^{1/d}$
\end{itemize}
If super-graph has emergent dimensionality $d$, diameter should scale like $N^{1/d}$. This tests whether the cloth has spatial structure vs just being well-connected.

\subsubsection*{Pass Criterion}
Clear scaling law ($R^2 > 0.90$), consistent with $d = 2$--4. Example: $D_{\max} \sim N^{1/3}$ suggests $d \approx 3$.

\subsubsection*{Implementation Notes}
\begin{itemize}
\item Use NetworkX \texttt{diameter()} or \texttt{eccentricity()} functions
\item If disconnected, compute diameter of giant component only
\item Average across seeds at each $N$
\end{itemize}

\section{Implementation: 3-Phase Plan (Revised)}

\subsection{Phase 1: Core Validation (Weeks 1--2)}

\textbf{Priority tests (revised based on methodological review):}
\begin{enumerate}
\item \textbf{Test 2.4: Ball-growth effective dimension} (most robust for small $K$, data already available)
\item \textbf{Test 3.1: Thread localization} (most important conceptual validation---establishes physics connection)
\item \textbf{Test 4.1: Community count scaling} (validates coarse-graining is occurring)
\item \textbf{Test 2.1: Spectral dimension} (OPTIONAL---informative but noisy at $N=8$)
\end{enumerate}

\textbf{Tests DEFERRED from Phase 1:}
\begin{itemize}
\item Test 2.2 (triangle inequality)---tautological for unweighted graphs; merge into Test 3.3 in Phase 3
\end{itemize}

\textbf{Datasets:}
\begin{itemize}
\item Use existing $N \in \{4, 8\}$, $n \in \{0.4, 0.8\}$, 5 seeds each
\item Add $N=16$ if time permits (from scaling benchmark)
\end{itemize}

\textbf{Deliverable:}
\begin{itemize}
\item Short report: ``Super-graph is geometric and physical''
\item 3 core measures: $d_{\text{eff}}$ (geometry), LOC (physics), $\alpha$ (coarse-graining)
\item Decision point: If all three pass, proceed to Phase 2. If Test 3.1 (thread localization) fails, major red flag---geometry may not be where physics lives.
\end{itemize}

\subsection{Phase 2: Extended Validation (Weeks 3--4)}

\textbf{Priority tests:}
\begin{enumerate}
\item Test 3.2: Cross-link density vs interaction (validates Path A mechanism)
\item Test 2.1: Spectral dimension at larger $N$ (if deferred from Phase 1)
\item Test 4.2: Spectral dimension convergence (requires multiple $N$ runs)
\end{enumerate}

\textbf{Datasets:}
\begin{itemize}
\item Extend to $N \in \{4, 8, 16, 32\}$ for scaling tests
\item Use 5--10 seeds per $N$ at $n \in \{0.6, 0.8\}$
\end{itemize}

\textbf{Deliverable:}
\begin{itemize}
\item Report: ``Path A mechanism connects cloth to micro-dynamics''
\item Evidence that cross-link strength predicts thread interactions
\item Decision point: If tests pass, Path A is validated. Proceed to full N-scan or Phase 3 deep validation.
\end{itemize}

\subsection{Phase 3: Advanced Validation (Weeks 5--6, Optional)}

\textbf{Deep tests (if Phase 1--2 pass):}
\begin{enumerate}
\item Test 2.3: Curvature distribution (claim geometric richness)
\item Test 3.3: Metric reconstruction (extract explicit $g_{\mu\nu}$, includes weighted triangle inequality check)
\item Test 4.3: Super-graph diameter scaling (spatial vs small-world)
\end{enumerate}

\textbf{Datasets:}
\begin{itemize}
\item Full $N$-scan: $N \in \{8, 16, 24, 32, 48, 64\}$ (stop before $N=128$ for now)
\item 10--30 seeds per $N$ depending on computational cost
\end{itemize}

\textbf{Deliverable:}
\begin{itemize}
\item Full validation report: ``Path A as geometric mechanism''
\item Complete evidence package for BCQM VI paper
\item Readiness for $N=128$ production runs and BCQM VII metric extraction
\end{itemize}

\section{Success Criteria by Strength}

\subsection{Minimal Viable Result (Sufficient for BCQM VI)}

\begin{itemize}
\item Effective dimension $d_{\text{eff}}$ from ball growth in physical range ($d_{\text{eff}} \in [1.5, 4.5]$, std dev $<15\%$)
\item Thread localization significantly above random (LOC $> 0.7$, $p < 0.001$)
\item Community count scaling sublinear: $K \sim N^\alpha$ with $\alpha < 0.8$
\end{itemize}

\textbf{Interpretation:} Path A creates stable coarse-grained structure with geometric properties that particles respect. Sufficient to publish as ``mechanism exploration'' paper.

\subsection{Strong Result (Supports BCQM VII Continuation)}

All minimal criteria, plus:
\begin{itemize}
\item Spectral dimension $d_s$ finite and stable at $N \geq 32$ ($d_s \in [1.5, 4]$, std dev $<10\%$)
\item Cross-link density predicts thread interaction: Spearman $\rho > 0.5$
\item Triangle inequality holds for weighted distances: $\eta > 0.95$
\end{itemize}

\textbf{Interpretation:} Super-graph is genuine emergent geometry encoding physical proximity. Strong justification for metric extraction in BCQM VII.

\subsection{Publication-Grade Result}

All strong criteria, plus:
\begin{itemize}
\item Non-trivial curvature distribution (not flat, stable across seeds)
\item Successful metric embedding: stress $< 0.15$, $d_{\text{embed}} \leq 4$
\item Clear scaling laws with $N$: $d_s$ converges, diameter scales as $N^{1/d}$
\end{itemize}

\textbf{Interpretation:} Path A generates emergent spacetime with identifiable dimensionality, curvature, and metric structure. Publication target: PRL, PRD, or similar high-impact venue.

\section{Timeline and Resource Requirements}

\subsection{Computational Resources}

\begin{itemize}
\item \textbf{Phase 1--2:} Can run on existing hardware (laptop or single Mac Mini)
\item \textbf{Phase 3:} May require Mac Mini cluster if extending to $N=64$
\item \textbf{Scaling benchmark:} Run before committing to Mac Mini purchase
\end{itemize}

\subsection{Timeline}

\begin{tabular}{ll}
\toprule
\textbf{Phase} & \textbf{Duration} \\
\midrule
Phase 1 (Core validation) & 2 weeks \\
Decision point 1 & 2 days \\
Phase 2 (Extended validation) & 2 weeks \\
Decision point 2 & 2 days \\
Phase 3 (Advanced validation, optional) & 2 weeks \\
Final report \& BCQM VI draft outline & 1 week \\
\midrule
\textbf{Total} & \textbf{6--9 weeks} \\
\bottomrule
\end{tabular}

\subsection{Decision Points}

\textbf{After Phase 1:}
\begin{itemize}
\item If core tests pass (geometry + physics + coarse-graining) $\rightarrow$ proceed to Phase 2
\item If Test 3.1 (thread localization) fails $\rightarrow$ \textbf{major concern}---super-graph may be geometric artifact not connected to physics; diagnose or consider pivot
\item If Test 2.4 or 4.1 fail $\rightarrow$ diagnose (implementation bug? wrong coarse-graining? Path A insufficient?)
\end{itemize}

\textbf{After Phase 2:}
\begin{itemize}
\item If extended tests pass $\rightarrow$ Path A validated, proceed to full N-scan or Phase 3 deep validation
\item If physics tests fail $\rightarrow$ super-graph is geometric but not physical; consider pivot to alternative cross-link mechanism
\end{itemize}

\section{Next Immediate Actions (Revised)}

\begin{enumerate}
\item \textbf{This weekend:} Implement Test 2.4 (ball-growth dimension extraction) on existing data---quick win, no new runs needed, most robust for small $K$
\item \textbf{Early next week:} Implement Test 3.1 (thread localization)---\textbf{highest priority conceptual validation}, uses existing thread + community data
\item \textbf{Mid next week:} Implement Test 4.1 (community scaling $K$ vs $N$)---quick check on coarse-graining using $N \in \{4, 8\}$ data
\item \textbf{Optional:} Implement Test 2.1 (spectral dimension)---informative but expect noisy results at $N=8$; do not treat as blocking
\item \textbf{Week 2:} Analyze Phase 1 results, decision point 1
\item \textbf{Week 3--4:} Implement Phase 2 tests (interactions, convergence)
\item \textbf{Week 5:} Decision point 2, plan full N-scan or Phase 3
\end{enumerate}

\section*{Conclusion}

This test plan (revised after methodological review) provides a systematic path to validate that the super-graph is not just stable (Gate 1, already passed), but genuinely geometric (Gate 2), physically meaningful (Gate 3), and scalable (Gate 4). 

\textbf{Key revisions:}
\begin{itemize}
\item Prioritized Test 3.1 (thread localization) as the conceptual linchpin---it directly tests whether particles live on the emergent geometry
\item Promoted Test 2.4 (ball-growth dimension) as the primary Phase 1 geometry diagnostic---more robust than spectral dimension at small $K$
\item Flagged Test 2.2 (triangle inequality) as tautological for unweighted graphs; deferred to Phase 3
\item Added finite-size warnings for Test 2.1 (spectral dimension) at $N=8$
\end{itemize}

Completion of Phase 1 (minimal viable result) establishes that Path A creates geometric structure that particles respect---sufficient to justify continuing and drafting BCQM VI. Completion of Phases 2--3 (publication-grade result) provides strong evidence for emergent spacetime and supports transition to BCQM VII metric extraction work.

\vspace{1em}
\noindent\textbf{Status:} Ready to begin Phase 1 implementation using existing $N \in \{4, 8\}$ data. Priority order: Test 2.4 $\rightarrow$ Test 3.1 $\rightarrow$ Test 4.1.

\end{document}
```

