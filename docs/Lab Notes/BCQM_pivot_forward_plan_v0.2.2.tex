\documentclass[11pt,a4paper]{article}

\usepackage[margin=1in]{geometry}
\usepackage[T1]{fontenc}
\usepackage[utf8]{inputenc}
\usepackage[british]{babel}
\usepackage{lmodern}
\usepackage{microtype}
\usepackage{csquotes}
\usepackage{amsmath,amssymb}
\usepackage{booktabs}
\usepackage{xurl}
\usepackage{hyperref}
\usepackage{enumitem}
\setlist{nosep}

\title{\textbf{BCQM VII Stage-2 (Geometry Cloth) Forward Plan}\\
\large Pivot Programme: From molecular edges to a stable ``cloth'' via bin-coarsening and community structure}
\author{Peter M.~Ferguson \\ \textit{Independent Researcher}}
\date{30 January 2026}

\begin{document}
\maketitle
\emergencystretch=2em
\sloppy
\section{Purpose of the pivot}
Stage--2 aims to define a persistent ``cloth'' geometry object beyond the short, fast-mixing active slice \(V_{\mathrm{active}}(t)\). Stage--1 established that exact edge identity is seed-sensitive (molecular/foam scale), while geometry diagnostics (ball growth) can be comparatively stable. The pivot operationalises this by moving the geometry object one level up: from raw edge sets to coarse-grained structure (communities and a community super-graph).

\subsection{Working diagnosis from Stage--2 experiments}
In the current VII code, permissive persistence (hits1) yields a connected edge cloth and highly stable ball-growth curves at high cross-link pressure, while strict cross-bin repetition (hits2) isolates tiny recurrent motifs rather than a spanning backbone (even at longer epochs and bin coarsening). This indicates that ``cloth'' stability should be sought at an intermediate, coarse-grained scale: stable \emph{classes} of geometry may emerge before stable microstructure.

\section{What must be fixed before proceeding}
This plan assumes the following are already in place (and verified):
\begin{itemize}
\item Proven code provenance: BCQM VII baseline cryptographically matched to BCQM VI v0.1.0 core files.
\item Stage--2 logging: \path{cloth} and \path{cloth_ledger} in \path{RUN_METRICS_*}. Explicit \path{core_edges_used} and \path{core_events_used} lists are recorded for survival.
\item Consolidation: ensemble outputs retained under \path{outputs_cloth/}; summary CSVs retained under a local \path{csv/} folder.
\end{itemize}

\section{Pivot hypothesis and success definition}
\subsection{Hypothesis}
If exact edges fluctuate (edge Jaccard \(\ll 1\)) but ball-growth geometry is stable, then a coarse-grained object --- communities on the cloth core and an inter-community super-graph --- should exhibit intermediate stability: greater than raw edges and robust enough to define ``thread scale'' structure.

\subsection{Success definition}
Stage--2 pivot succeeds if, in a fixed regime (start with \(N=8\), \(n=0.8\), \(W_{\mathrm{coh}}=100\) on the hits1 connected cloth baseline):
\begin{itemize}
\item community partitions show moderate-to-strong agreement across seeds (NMI/ARI),
\item the community super-graph is more stable than raw edge sets,
\item super-graph geometry diagnostics (ball growth) are at least as stable as the current cloth ball-growth curves,
\item and these conclusions survive modest variations (seeds, epoch length, binning).
\end{itemize}

\section{Operational definition of the base graph for coarse-graining}
\subsection{Base adjacency}
Use the \textbf{hits1 used-by-core cloth core} as the base object: the directed edge set \path{core_edges_used} and its induced event set \path{core_events_used}. For community detection, build the undirected projection of this edge set (treat each directed edge as an undirected adjacency).

\subsection{Why hits1 baseline}
hits1 yields a connected cloth and very stable ball-growth metrics at high \(n\). It provides the ``bed-sheet'' substrate on which communities can be detected. hits2 is retained as a motif detector and robustness probe, but is not used as the primary base graph at this stage.

\section{Test 4A: Community-structure stability}
\subsection{Parameters}
\begin{itemize}
\item Fixed regime: \(N = 8\), \(n = 0.8\), \(W_{\mathrm{coh}} = 100\), hits1 baseline.
\item Seeds: 5--10 (use existing checkpoint ensembles if available).
\item Epoch: start with the x10, bins=20 baseline (stable ball-growth regime), then vary.
\end{itemize}

\subsection{Procedure}
For each seed \(s\):
\begin{enumerate}
\item Extract the cloth core edge set (hits1): \texttt{core\_edges\_used}.
\item Build undirected projection \(G^{\mathrm{core}}_s\) for community detection.
\item Apply community detection:
\begin{itemize}
\item Primary: Louvain modularity (fast, standard).
\item Alternative: Leiden (higher quality), or spectral clustering for comparison.
\end{itemize}
\item Record the partition vector \(\mathbf{c}_s\): mapping from event to community ID, and record community count \(K_s\).
\end{enumerate}

\subsection{Partition similarity observables}
For each pair of seeds \((s,s')\), compute:
\begin{itemize}
\item Normalised Mutual Information (NMI) between \(\mathbf{c}_s\) and \(\mathbf{c}_{s'}\).
\item Adjusted Rand Index (ARI) between \(\mathbf{c}_s\) and \(\mathbf{c}_{s'}\).
\item Community-count consistency: distribution of \(K_s\) across seeds.
\end{itemize}
Interpretation guide: NMI \(\approx 1\) identical; NMI \(\approx 0\) random; NMI \(>0.5\) moderate; \(>0.7\) strong.

\subsection{Super-graph stability observables}
For each seed \(s\), build a directed community super-graph \(G^{\mathrm{super}}_s\):
\begin{itemize}
\item nodes: communities from \(\mathbf{c}_s\),
\item edges: \(C_i\to C_j\) exists if any cloth-core edge \((u\to v)\) crosses from \(u\in C_i\) to \(v\in C_j\),
\item weights: number of crossing edges (aggregate flow).
\end{itemize}

Across seed pairs:
\begin{itemize}
\item Super-graph edge Jaccard (after label alignment).
\item Weighted flow correlation (Pearson correlation on common edges).
\item Super-graph ball growth: compare \(|B(r)|/|C|\) curves across seeds via L2 distance.
\end{itemize}

\subsection{Success criteria (Test 4A)}
\begin{itemize}
\item Mean NMI \(> 0.5\) (moderate) or \(> 0.7\) (strong).
\item Super-graph edge Jaccard \(\gg\) raw edge Jaccard (e.g.\ \(> 0.2\) vs \(\approx 0.01\)).
\item Super-graph ball-growth L2 distances \(\lesssim\) cloth ball-growth distances on the hits1 core.
\end{itemize}

\section{Test 4B: Event-filtered edge core}
\subsection{Rationale}
Persistent events are often more stable than exact edges. Filter edges to those whose endpoints are persistent events, suppressing seed-specific micro-loops.

\subsection{Procedure}
For each seed \(s\):
\begin{enumerate}
\item Identify persistent events \(E^{\mathrm{persist}}_s\) using event occupancy (bin hits / number of bins), starting with a permissive threshold.
\item Define event-filtered core edges: edges \((u\to v)\) with \(u,v\in E^{\mathrm{persist}}_s\) and used by the core.
\end{enumerate}

\subsection{Observables and success criteria}
\begin{itemize}
\item Edge Jaccard across seeds improves relative to raw hits1 edges.
\item Cloth remains connected (no severe fragmentation).
\item Ball-growth stability is maintained or improved.
\end{itemize}

\section{Test 4C: Quantile-based edge persistence}
\subsection{Rationale}
Instead of requiring explicit cross-bin repetition (hits2), keep the top \(q\%\) most-used core edges over an epoch (high-traffic backbone). This is a coarse-grained persistence proxy without mutating primitives.

\subsection{Procedure}
For each seed \(s\):
\begin{enumerate}
\item Compute per-edge occupancy \(p_{\mathrm{persist}}(u\to v)=\text{bin\_hits}(u\to v)/B\) from the run ledger.
\item Keep edges above a quantile threshold (e.g.\ top 10\%, 20\%, 30\%).
\item Evaluate connectivity and geometry on this quantile cloth.
\end{enumerate}

\subsection{Success criteria}
\begin{itemize}
\item Edge Jaccard increases as the quantile becomes stricter (top 10\% more stable than top 30\%).
\item Quantile cloth remains connected at intermediate thresholds (e.g.\ 20\%).
\item Ball-growth stability maintained.
\end{itemize}

\section{Implementation notes}
\subsection{Packages}
Community detection can be implemented with \texttt{python-louvain} (Louvain) or \texttt{leidenalg} (Leiden), with graph handling in \texttt{networkx} or \texttt{igraph}. NMI and ARI can be computed with \texttt{sklearn.metrics}. For super-graph label alignment use maximum-weight bipartite matching (Hungarian algorithm; \texttt{scipy.optimize.linear\_sum\_assignment}).

\subsection{Directed versus undirected handling}
Community detection is performed on an undirected projection of the cloth core. The resulting community labels are then used to build a directed super-graph for flow stability analysis.

\subsection{No new runs requirement}
Test 4A and 4B can be performed offline on existing hits1 ensembles (recommended starting point). Test 4C requires per-edge occupancy counts; if these are not logged explicitly, they can be reconstructed from stored ledgers.

\section{Gated execution plan}
\begin{enumerate}[label=\arabic*.]
\item Gate 0: Confirm hits1 cloth baseline (connected core; stable ball-growth) in the fixed regime.
\item Gate 1: Community partitions are stable (NMI/ARI thresholds achieved).
\item Gate 2: Super-graph stability exceeds raw edge stability.
\item Gate 3: Geometry diagnostics on the super-graph are stable across seeds.
\item Gate 4: Only then attempt additional ``physics'' proxies (e.g.\ causal-frontier definitions on cloth) if desired.
\end{enumerate}

\section{Deliverables}
\begin{itemize}
\item A reproducible analysis script that, given an ensemble folder, outputs: NMI/ARI matrices, super-graph stability tables, and super-graph ball-growth curves.
\item A compact CSV summary similar to the existing survival tables, but at the community and super-graph level.
\item A short checkpoint lab note documenting whether the pivot succeeded at Gate 1/2/3.
\end{itemize}

\end{document}
