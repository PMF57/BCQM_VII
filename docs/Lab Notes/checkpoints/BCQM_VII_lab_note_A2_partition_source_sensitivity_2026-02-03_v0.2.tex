\documentclass[11pt]{article}

\usepackage[a4paper,margin=25mm]{geometry}
\usepackage[T1]{fontenc}
\usepackage[utf8]{inputenc}
\usepackage[british]{babel}
\usepackage{lmodern}
\usepackage{microtype}
\usepackage{csquotes}
\usepackage{amsmath,amssymb}
\usepackage{booktabs}
\usepackage{xurl}
\usepackage{hyperref}
\usepackage[nameinlink,noabbrev]{cleveref}
\usepackage{enumitem}
\setlist{nosep}

\title{BCQM VII Lab Note\\A2: Partition-source sensitivity (core/core vs all/all) (v0.1)}
\author{Peter M.~Ferguson \\ \textit{Independent Researcher}}
\date{3 February 2026}

\begin{document}
\emergencystretch=2em
\sloppy
\maketitle

\section*{Purpose}
Record the A2 ``stake-in-the-ground'' sensitivity test: determine how the Stage--2 pivot metrics (Gates 1--3) depend on whether the community partition and super-graph are constructed from (i) \textbf{core-only} cloth edges (core/core) or (ii) \textbf{all-used} edges (core+halo; all/all). This resolves an ambiguity exposed by Gate--4 localisation at larger \(N\): for trajectories, all-used sources can be needed for coverage, but for geometry it may not be desirable.

\section*{Data and scripts used}
\paragraph{Dataset.}
The baseline ensemble used was the Stage--2 pivot dataset:
\begin{itemize}
\item hits1, x10 epoch, bins=20, \(W_{\mathrm{coh}}=100\), seeds 56791--56795
\item \(N\in\{4,8\}\), \(n\in\{0.4,0.8\}\)
\item run folder: \path{outputs_cloth/ensemble_W100_N4N8_hits1_x10_bins20}
\end{itemize}

\paragraph{CSV inputs (A2.zip).}
This lab note is based on two output folders (zipped for transfer):
\begin{itemize}
\item \path{csv/A2_core_core/}:
  \begin{itemize}
  \item \path{community_partition_stability.csv}
  \item \path{supergraph_edge_stability.csv}
  \item \path{supergraph_ballgrowth_stability.csv}
  \end{itemize}
\item \path{csv/A2_all_all/}:
  \begin{itemize}
  \item \path{community_partition_stability.csv}
  \item \path{supergraph_edge_stability.csv}
  \item \path{supergraph_ballgrowth_stability.csv}
  \end{itemize}
\end{itemize}

\paragraph{Scripts.}
A2 used the A2-enabled analysis scripts (no simulation reruns). The A2 variants add an \texttt{edge\_source} switch to select either core-only edges or core+halo (all-used) edges:
\begin{itemize}
\item \path{bcqm_vii_cloth/analysis/community_cloth_stability_A2.py} (adds \texttt{--edge\_source core} or \texttt{--edge\_source all})
\item \path{bcqm_vii_cloth/analysis/supergraph_edge_stability_A2.py} (adds \texttt{--edge\_source core} or \texttt{--edge\_source all})
\item \path{bcqm_vii_cloth/analysis/community_cloth_stability.py} (super-graph ball-growth stability from the same dataset; unchanged script)
\end{itemize}

\section*{Summary of results}
Table~\ref{tab:a2} summarises the key deltas between core/core and all/all, quadrant by quadrant. Quantities are mean values across seed pairs (5 seeds \(\Rightarrow\) 10 pairs).

\begin{table}[ht]
\centering
\small
\begin{tabular}{@{}cccccccc@{}}
\toprule
\(N\) & \(n\) &
\(K_{\mathrm{core}}\) & \(K_{\mathrm{all}}\) &
\(\mathrm{NMI}_{\mathrm{core}}\) & \(\mathrm{NMI}_{\mathrm{all}}\) &
\(d_{\mathrm{L2,core}}\) & \(d_{\mathrm{L2,all}}\) \\
\midrule
4 & 0.4 & 44.2 & 24.4 & 0.833 & 0.810 & 0.0219 & 0.0545 \\
4 & 0.8 & 22.4 & 22.4 & 0.834 & 0.834 & 0.0170 & 0.0170 \\
8 & 0.4 & 75.4 & 23.8 & 0.838 & 0.822 & 0.0085 & 0.0162 \\
8 & 0.8 & 22.4 & 22.4 & 0.835 & 0.835 & 0.0235 & 0.0235 \\
\bottomrule
\end{tabular}
\caption{A2 sensitivity summary: core/core vs all/all. Here \(K\) is the mean community count, NMI is partition similarity across seeds, and \(d_{\mathrm{L2}}\) is the mean L2 distance between normalised super-graph ball-growth curves across seeds. Full CSVs are listed in the text.}
\label{tab:a2}
\end{table}

\paragraph{Gate 2 invariance.}
Super-graph edge stability (unweighted Jaccard) and weight correlations are unchanged between core/core and all/all for this dataset. That is, including halo edges does not change the set of community-to-community connections or their relative flow strengths as measured by Gate 2. This indicates that the stable mesoscopic flow skeleton is already captured by the core edges alone.

\paragraph{Low-\(n\) sensitivity (n=0.4).}
At low cross-link pressure, the choice of edge source matters:
\begin{itemize}
\item The all/all construction collapses the super-graph to a much smaller connected component (mean component size drops to \(\sim 24\) communities for \(N=8\)), and the geometry diagnostic becomes noisier (ball-growth stability degrades).
\item The core/core construction yields a larger, richer super-graph (higher \(K\) and larger GCC) and more stable ball-growth curves across seeds.
\end{itemize}
Interpretation: at \(n=0.4\) the halo edges act primarily as ``closing'' edges that reduce fragmentation and can short-cut the community graph, but this does not improve reproducible geometry; instead it increases variability in the derived super-graph ball-growth curves.

\paragraph{High-\(n\) robustness (n=0.8).}
At high cross-link pressure, core/core and all/all become equivalent: \(K\), NMI/ARI, super-graph edge stability, and super-graph ball-growth stability are the same. This is consistent with the high-\(n\) regime being dominated by the coherent core, with relatively little independent halo structure.

\section*{Methodological conclusion (recommended defaults)}
A2 resolves the ``which graph?'' ambiguity into a clean practice:
\begin{itemize}
\item \textbf{Gates 1--3 (geometry object definition):} use \textbf{core/core} by default, especially at low \(n\), because it preserves a richer super-graph and yields more stable super-graph geometry diagnostics.
\item \textbf{Gate 4 (trajectory/localisation tests):} use \textbf{all/all} by default when needed for coverage, and explicitly report the mapping coverage fraction. This is particularly important at larger \(N\) or weaker regimes where core-only partitions may not cover the traced event IDs.
\end{itemize}

\section*{Next steps}
\begin{enumerate}[label=\arabic*.]
\item Incorporate this A2 conclusion into the Stage--2 pivot record and the ``Beyond Stability'' plan as a methodological note.
\item For publication clarity, add a short ``graph-choice'' subsection explaining why core/core is the geometry object and all/all is the localisation substrate (coverage).
\item Optional: repeat A2 at N=16 or N=32 if later needed, but the key qualitative behaviour is already established here.
\end{enumerate}

\end{document}
