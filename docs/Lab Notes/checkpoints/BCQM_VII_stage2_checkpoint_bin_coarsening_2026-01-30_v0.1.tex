\documentclass[11pt]{article}

\usepackage[a4paper,margin=25mm]{geometry}
\usepackage[T1]{fontenc}
\usepackage[utf8]{inputenc}
\usepackage[british]{babel}
\usepackage{lmodern}
\usepackage{microtype}
\usepackage{csquotes}
\usepackage{amsmath}
\usepackage{amssymb}
\usepackage{xurl}
\usepackage{hyperref}
\usepackage[nameinlink,noabbrev]{cleveref}
\usepackage{enumitem}
\setlist{nosep}

\title{BCQM VII Stage--2 Checkpoint Lab Note\\Bin coarsening test for strict persistence (hits2) at long epochs (v0.1)}
\author{Peter M.~Ferguson \\ \textit{Independent Researcher}}
\date{30 January 2026}

\begin{document}
\emergencystretch=2em
\sloppy
\maketitle

\section*{Purpose}
This checkpoint tests whether the failure of strict persistence (hits2) to yield a connected edge-cloth core is primarily a \emph{binning artefact}. The hypothesis is that coarsening bins (fewer, longer bins) may increase cross-bin repetition and allow a hits2 cloth to span, while keeping the underlying dynamics unchanged. We also include a calibration run (hits1) to confirm the expected ``connected cloth'' regime at the same long epoch length.

\section*{Context (prior Stage--2 checkpoint)}
In the initial Stage--2 cloth work:
\begin{itemize}
\item \textbf{hits1} (\texttt{min\_bin\_hits}=1) produced a connected edge-cloth core and stable ball-growth geometry across seeds, but exact edge identity remained seed-sensitive.
\item \textbf{hits2} (\texttt{min\_bin\_hits}=2) tended to collapse to tiny recurrent pockets (or empties), behaving more like a motif detector than a cloth backbone at the tested epoch lengths.
\end{itemize}
This checkpoint asks whether that behaviour is due to ``time chopped too finely'' (bins) rather than the persistence definition itself.

\section*{Runs performed}
All runs used the same long epoch length (x10 of the Stage--2 baseline), \(W_{\mathrm{coh}}=100\), 5 seeds per quadrant, and the same scan grid:
\begin{itemize}
\item \(N\in\{4,8\}\), \(n\in\{0.4,0.8\}\)
\item lockstep support threshold \(w_{\mathrm{lock}}=0.10\)
\item concurrency diagnostic threshold \texttt{min\_concurrency}=2
\end{itemize}
We varied only the bin count and \texttt{min\_bin\_hits}:
\begin{itemize}
\item hits2, x10, \textbf{bins=20}
\item hits2, x10, \textbf{bins=10}
\item calibration: hits1, x10, \textbf{bins=20}
\end{itemize}
Outputs were analysed by set survival (Jaccard on \texttt{core\_edges\_used} and \texttt{core\_events\_used}) and metric survival (L2 distance between normalised ball-growth curves \(|B(r)|/|C|\) across seeds).

\section*{Results}
\subsection*{1. hits2 remains a motif detector under bin coarsening}
Coarsening bins from 20 to 10 did not convert hits2 into a connected cloth backbone:
\begin{itemize}
\item For \(n=0.8\), hits2 edge cores remain small, with ball-growth component sizes \(\approx 2.4\)--\(3\) on average, and edge-set Jaccard across seeds remains zero in the non-degenerate regimes.
\item For \(n=0.4\), hits2 often yields empty or near-empty edge cores; any Jaccard values near 1 are therefore dominated by ``identical emptiness'' rather than stable structure.
\end{itemize}
Thus, the failure of hits2 to yield a spanning cloth at this scale is not primarily a binning artefact.

\subsection*{2. Calibration (hits1, x10, bins=20) confirms the connected cloth regime}
The calibration run (hits1 at the same long epoch length) produced large connected cloth components and very stable geometry diagnostics, especially in the high-cross-link regime:
\begin{itemize}
\item Event-core survival at \(n=0.8\) is near 1 (Jaccard \(\approx 0.98\) for both \(N=8\) and \(N=4\)).
\item Ball-growth metric survival is extremely tight (cross-seed L2 distances \(\ll 10^{-3}\) in the \(n=0.8\) regime).
\item Exact edge-set survival remains small (as previously), indicating that stability appears first at the metric/diagnostic level rather than at exact edge identity.
\end{itemize}

\section*{Interpretation}
The bin coarsening test supports the ``channel activity versus cloth'' hierarchy:
\begin{itemize}
\item bin-level concurrency captures short-run channelisation,
\item strict cross-bin repetition (hits2) isolates recurrent motifs,
\item connected cloth geometry emerges in the permissive regime (hits1) and is stable as a geometry class even when microstructure varies.
\end{itemize}
Therefore, if a stricter-than-hits1 edge cloth is required, the next step is to change the persistence definition rather than to further adjust binning.

\section*{Immediate next options}
Two minimalism-preserving edge persistence definitions are now prioritised for testing:
\begin{enumerate}[label=\arabic*.]
\item \textbf{Event-filtered edge core:} define persistent events first, then keep edges between persistent events (used by lockstep core), which may suppress seed-specific micro-loops.
\item \textbf{Quantile persistence:} keep the top \(q\)\% most-used core edges over the epoch (rather than requiring cross-bin repetition), to obtain a stable backbone class without enforcing exact repeatability.
\end{enumerate}

\section*{Files}
This checkpoint corresponds to the consolidated summary table:
\begin{itemize}
\item \texttt{cloth\_bin\_coarsening\_summary\_W100\_x10.csv}
\end{itemize}

\end{document}
