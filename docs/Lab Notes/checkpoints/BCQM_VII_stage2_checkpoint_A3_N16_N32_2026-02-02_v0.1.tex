\documentclass[11pt]{article}

\usepackage[a4paper,margin=25mm]{geometry}
\usepackage[T1]{fontenc}
\usepackage[utf8]{inputenc}
\usepackage[british]{babel}
\usepackage{lmodern}
\usepackage{microtype}
\usepackage{csquotes}
\usepackage{amsmath,amssymb}
\usepackage{booktabs}
\usepackage{xurl}
\usepackage{hyperref}
\usepackage[nameinlink,noabbrev]{cleveref}
\usepackage{enumitem}
\setlist{nosep}

\title{BCQM VII Stage--2 Checkpoint\\A3 Scale-Up: N=16 and N=32 stress tests (v0.1)}
\author{Peter M.~Ferguson \\ \textit{Independent Researcher}}
\date{2 February 2026}

\begin{document}
\emergencystretch=2em
\sloppy
\maketitle

\section*{Purpose}
This checkpoint consolidates the A3 scale-up runs at \(N=16\) and \(N=32\) (hits1 cloth, x10 epoch, bins=20, \(n=0.8\), 5 seeds). The goal is to test whether the Stage--2 pivot conclusions---stable mesoscopic ``cloth'' communities, stable super-graph structure, and thread locality on the super-graph---persist beyond the small-\(N\) regime.

\section*{Run definitions}
\paragraph{Common settings.}
Both scale-up runs use the same regime:
\begin{itemize}
\item Cloth baseline: hits1 (\texttt{min\_bin\_hits}=1), \(W_{\mathrm{coh}}=100\), bins=20, x10 epoch.
\item Cross-link pressure: \(n=0.8\).
\item Seeds: 56791--56795.
\item Logging: \texttt{store\_lists=true} to record \path{core_edges_used} and \path{core_events_used}; and \texttt{trace\_threads=true} with \texttt{trace\_stride=1} to produce \path{cloth_trace} (per-bin per-thread end-of-bin event IDs and core-mask).
\end{itemize}

\paragraph{Outputs.}
\begin{itemize}
\item Run folders: \path{outputs_cloth/gateA3_N16_hits1_x10_bins20_n0p8/} and \path{outputs_cloth/gateA3_N32_hits1_x10_bins20_n0p8/}.
\item Summary CSVs (generated locally): \path{csv/gateA3_N16/*} and \path{csv/gateA3_N32/*}, including run summaries and ball-growth pairwise stability tables.
\end{itemize}

\section*{N=16 results (high-level)}
\subsection*{Run health}
The \(N=16\) runs are well-behaved: the cloth core is large and connected, trace logging is enabled for all seeds, and the Stage--1 connectivity indicators remain in-family (``space on'').

\subsection*{Cloth geometry stability}
Across the 5 seeds, cloth ball-growth fraction curves \(|B(r)|/|C|\) are extremely stable (pairwise L2 distances of order \(10^{-4}\)), indicating that the geometry class remains tightly reproducible at \(N=16\).

\subsection*{Interpretation}
At \(N=16\), the high-\(n\) regime remains close to ``all core'': halo counts are small compared to the core. This is consistent with a dominant coherent population and provides a clean baseline for community and super-graph tests.

\section*{N=32 results (high-level)}
\subsection*{Run health}
The \(N=32\) runs are also stable and in-family (``space on''; trace logging present; no pathological warnings). The core cloth is connected and diagnostic-stable, but the core/halo decomposition becomes nontrivial.

\subsection*{Core/halo structure}
Unlike \(N=16\), at \(N=32\) the halo contribution is substantial (large persistent event and edge counts outside the dominant core). This is a desirable property for Stage--2: it places the model in a more realistic regime where a stable core background coexists with a significant ambient population.

\subsection*{Cloth geometry stability}
Even with a large halo, the cloth geometry diagnostic remains stable: pairwise L2 distances between normalised ball-growth curves remain small (order \(10^{-3}\)). This supports the ``stability at the level of geometry class, not microstructure'' interpretation at larger N.

\section*{Gate--4 localisation at N=32 (physics proxy)}
\subsection*{All-used partition and super-graph (recommended)}
Using \texttt{partition\_source=all} and \texttt{supergraph\_source=all} yields full mapping coverage (coverage \(=1\), missing labels \(=0\), no infinite distances). In this regime, hop-distance motion on the super-graph remains strongly local:
\begin{itemize}
\item all inter-bin transitions lie in \(d\in\{0,1,2\}\) with \(P(d\ge 3)=0\),
\item probability mass concentrates at \(d=0\) and \(d=1\) (dominant local motion),
\item core versus halo hop distributions are similar, with halo at most slightly ``rougher'' (small increase in \(d=2\) mass).
\end{itemize}
Thus, even at \(N=32\) with a nontrivial halo, threads respect the emergent coarse geometry.

\subsection*{Core-only partition/super-graph (diagnostic caveat)}
Using \texttt{partition\_source=core} and \texttt{supergraph\_source=core} yields very low mapping coverage (about 8\%), with hundreds of unmapped transitions per seed. Hop statistics in this mode are therefore computed on a small and biased subset of transitions and should not be used for physical claims. This is a methodological result: localisation tests require a partition/super-graph that covers the traced event IDs, which at larger N and nontrivial halo implies an all-used construction.

\section*{Interpretation and stake-in-the-ground}
These A3 scale-ups support the Stage--2 pivot claims beyond the small-\(N\) regime:
\begin{itemize}
\item The hits1 cloth baseline remains connected and geometry-diagnostic stable at \(N=16\) and \(N=32\).
\item At \(N=32\), a large halo emerges naturally while cloth-level diagnostics remain stable, reinforcing the ``core+ambience'' interpretation.
\item Gate--4 localisation (threads move locally on the super-graph) holds in the high-\(n\) regime at \(N=32\) when the partition and super-graph are built from all-used edges/flows.
\end{itemize}
In short: scale-up does not break the pivot; instead it strengthens it by moving from near-all-core behaviour (N=16) to a regime with a meaningful ambient population (N=32) while preserving stable coarse geometry and local motion on the emergent substrate.

\section*{Next steps}
\begin{enumerate}[label=\arabic*.]
\item Optional algorithm-family cross-check (Leiden) for Gate 1--3 if required.
\item Partition-source sensitivity formalisation: compare Gate 1--3 outputs for \texttt{partition\_source=core} vs \texttt{all} in the low-\(n\) regime.
\item Extend localisation tests to \(n=0.4\) at higher N if desired, using all-used partitions for full coverage and meaningful core/halo comparisons.
\end{enumerate}

\end{document}
